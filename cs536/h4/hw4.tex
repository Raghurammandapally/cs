\documentclass{article}
\usepackage[utf8]{inputenc}
\usepackage{forest}
\usepackage{graphicx}
\usepackage{fullpage}
\usepackage{amsmath}
\usepackage{syntax}
\setlength{\grammarindent}{8em} % increase separation between LHS/RHS 

\usepackage[letterpaper, margin=1in]{geometry}
\usepackage{listings}


% Top and Bottom Margin:  1  1/2"; Right and Left Margin:  1  1/2"
%\setlength{\topmargin}{0in}
%\setlength{\oddsidemargin}{0.5in}
%\setlength{\textwidth}{5.5in}
%\pagestyle{empty}
%\setlength{\parskip}{0.2in}

\title{CS536: Homework 4}
\author{Keith Funkhouser }
\date{October 6th, 2015}

\begin{document}

\maketitle


% *********************************
% TODO: translate from HTML
% *********************************
%<p class="reg">
%For this homework you will define a syntax-directed translation
%for the CFG given below, which defines a very simple programming language.
%</p>
%
%<dir>
%<pre>program &#8594; MAIN LPAREN RPAREN LCURLY list RCURLY
%
%list &#8594; list oneItem
     %| epsilon
      %
%oneItem &#8594; decl
        %| stmt
        %
%decl &#8594; BOOL ID SEMICOLON
     %| INT ID SEMICOLON
          %
%stmt &#8594; ID ASSIGN exp SEMICOLON 
     %| IF LPAREN exp RPAREN stmt
     %| LCURLY list RCURLY 
%
%exp &#8594; exp PLUS exp
%   | exp LESS exp
%   | exp EQUALS exp
%   | ID
%   | BOOLLITERAL
%   | INTLITERAL
%</pre></dir>


\section{}

Write a syntax-directed translation for the CFG given above, so that the
translation of an input program is the \emph{set} of variables used
somewhere in the program. Note: here the term \emph{use} is in contrast
to \emph{declaration}; any appearance of a variable that is not a
variable declaration counts as a \emph{use} of that variable. For the
example code in
\href{http://pages.cs.wisc.edu/~aws/courses/cs536-f15/asn/h4/h4.html\#Question2}{Question
2}, the translation should be \texttt{\{\ x,\ a,\ b\ \}}.

Your translation rules should use the following notation:

\begin{itemize}
\tightlist
\item
  \texttt{\{\ \}} is an empty set
\item
  \texttt{\{\ ID.value\ \}} is a set containing the variable whose name
  is the value associated with this \texttt{ID} token
\item
  \texttt{S1} \cap \texttt{S2} is the intersection of sets \texttt{S1} and
  \texttt{S2}
\item
  \texttt{S1} \cup \texttt{S2} is the union of sets \texttt{S1} and
  \texttt{S2}
\item
  \texttt{S1\ -\ S2} is the set of all items that are in \texttt{S1} but
  not in \texttt{S2}
\end{itemize}

Note that you should not try to use something like
"\texttt{\{\ a,\ b\ \}}" to mean a set with two elements; instead, use
set union to combine two sets that each contain one element.

Use the notation that was used in class and in the on-line readings;
i.e., use \texttt{nonterminal.trans} to mean the translation of a
nonterminal, and \texttt{terminal.value} to mean the value of a
terminal. Assume that \texttt{ID.value} is a \texttt{String} (the name
of the identifier). Use subscripts for translation rules that include
the same nonterminal or the same terminal more than once.

\section{}

Draw a parse tree for the program given below and annotate each
nonterminal in the tree with its translation.

\begin{verbatim}
main ( ) { 
    int x;
    bool y;
    x = a;
    int z;
    if (x == a) {
        int x;
        b = x < 18;
    }
}
\end{verbatim}

\begin{center}
\begin{forest}

  [program\\ \fbox{ \texttt{\{ a, b, x \} }}, align=center, base=bottom
    [MAIN\\ ``\texttt{main}'', align=center, base=bottom]
    [LPAREN\\ ``\texttt{(}'', align=center, base=bottom]
    [RPAREN\\ ``\texttt{)}'', align=center, base=bottom]
    [LCURLY\\ ``\texttt{\{}'', align=center, base=bottom]
    [list\\ \fbox{ \texttt{\{ a, b, x \} }}, align=center, base=bottom
      [list\\ \fbox{ \texttt{\{ a, x \} }}, align=center, base=bottom
        [list\\ \fbox{ \texttt{\{ x \} }}, align=center, base=bottom
          [list\\ \fbox{ \texttt{\{ x \} }}, align=center, base=bottom
            [list\\ \fbox{ \texttt{\{ x \} }}, align=center, base=bottom
              [list\\ \fbox{ \texttt{\{  \} }}, align=center, base=bottom
                [epsilon\\ $\epsilon$, align=center, base=bottom]
              ]
              [oneItem\\ \fbox{ \texttt{\{  \} }}, align=center, base=bottom
                [decl\\ \fbox{ \texttt{ \{ \} } }, align=center, base=bottom
                  [INT\\ ``\texttt{int}'', align=center, base=bottom]
                  [ID\\ ``\texttt{x}'', align=center, base=bottom]
                  [SEMICOLON\\ ``\texttt{;}'', align=center, base=bottom]
                ]
              ]
            ]
            [oneItem\\ \fbox{ \texttt{\{ x \} }}, align=center, base=bottom
              [decl\\ \fbox{ \texttt{ \{ \} } }, align=center, base=bottom
                [BOOL\\ ``\texttt{bool}'', align=center, base=bottom]
                [ID\\ ``\texttt{y}'', align=center, base=bottom]
                [SEMICOLON\\ ``\texttt{;}'', align=center, base=bottom]
              ]
            ]
          ]
          [oneItem\\ \fbox{ \texttt{\{ a, x \} }}, align=center, base=bottom
            [stmt\\ \fbox{ \texttt{ \{ a, x \} } }, align=center, base=bottom
              [ID\\ ``\texttt{x}'', align=center, base=bottom]
              [ASSIGN\\ ``\texttt{=}'', align=center, base=bottom]
              [exp\\ \fbox{ \texttt{ \{ a \} }}, align=center, base=bottom
                [ID\\ ``\texttt{a}'', align=center, base=bottom]
              ]
              [SEMICOLON\\ ``\texttt{;}'', align=center, base=bottom]
            ]
          ]
        ]
        [oneItem\\ \fbox{ \texttt{\{  \} }}, align=center, base=bottom
          [decl\\ \fbox{ \texttt{ \{ \} } }, align=center, base=bottom
            [BOOL\\ ``\texttt{bool}'', align=center, base=bottom]
            [ID\\ ``\texttt{z}'', align=center, base=bottom]
            [SEMICOLON\\ ``\texttt{;}'', align=center, base=bottom]
          ]
        ]
      ]
      [oneItem\\ \fbox{ \texttt{\{ a, b, x \} }}, align=center, base=bottom
        [stmt\\ \fbox{ \texttt{ \{ a, b, x \} } }, align=center, base=bottom
          [IF\\ ``\texttt{if}'', align=center, base=bottom]
          [LPAREN\\ ``\texttt{(}'', align=center, base=bottom]
          [exp\\ \fbox{ \texttt{ \{ a, x \} } }, align=center, base=bottom
            [exp\\ \fbox{ \texttt{ \{ x \} } }, align=center, base=bottom
              [ID\\ ``\texttt{x}'', align=center, base=bottom]
            ]
            [EQUALS\\ ``\texttt{=}'', align=center, base=bottom]
            [exp\\ \fbox{ \texttt{ \{ a \} } }, align=center, base=bottom
              [ID\\ ``\texttt{a}'', align=center, base=bottom]
            ]
          ]
          [RPAREN\\ ``\texttt{)}'', align=center, base=bottom]
          [stmt\\ \fbox{ \texttt{ \{ b, x \} } }, align=center, base=bottom
            [LCURLY\\ ``\texttt{\{}'', align=center, base=bottom]
            [list\\ \fbox{ \texttt{\{ b, x \} }}, align=center, base=bottom
              [list\\ \fbox{ \texttt{\{  \} }}, align=center, base=bottom
                [list\\ \fbox{ \texttt{\{  \} }}, align=center, base=bottom
                  [epsilon\\ $\epsilon$, align=center, base=bottom]
                ]
                [oneItem\\ \fbox{ \texttt{\{  \} }}, align=center, base=bottom
                  [decl\\ \fbox{ \texttt{ \{ \} } }, align=center, base=bottom
                    [INT\\ ``\texttt{int}'', align=center, base=bottom]
                    [ID\\ ``\texttt{x}'', align=center, base=bottom]
                    [SEMICOLON\\ ``\texttt{;}'', align=center, base=bottom]
                  ]
                ]
              ]
              [oneItem\\ \fbox{ \texttt{\{ b, x \} }}, align=center, base=bottom
                [stmt\\ \fbox{ \texttt{ \{ b, x \} } }, align=center, base=bottom
                  [ID\\ ``\texttt{b}'', align=center, base=bottom]
                  [ASSIGN\\ ``\texttt{=}'', align=center, base=bottom]
                  [exp\\ \fbox{ \texttt{ \{ x \} }}, align=center, base=bottom
                    [exp\\ \fbox{ \texttt{ \{ x \} } }, align=center, base=bottom
                      [ID\\ ``\texttt{x}'', align=center, base=bottom]
                    ]
                    [LESS\\ ``\texttt{<}'', align=center, base=bottom]
                    [exp\\ \fbox{ \texttt{ \{  \} } }, align=center, base=bottom
                      [INTLIT\\ ``\texttt{18}'', align=center, base=bottom]
                    ]
                  ]
                  [SEMICOLON\\ ``\texttt{;}'', align=center, base=bottom]
                ]
              ]
            ]
            [RCURLY\\ ``\texttt{\}}'', align=center, base=bottom]
          ]
        ]
      ]
    ]
    [RCURLY\\ ``\texttt{\}}'', align=center, base=bottom]
  ]

  [SEMICOLON\\ ``\texttt{;}'', align=center, base=bottom]
  [INTLIT\\ ``\texttt{val}'', align=center, base=bottom]
  [ID\\ ``\texttt{val}'', align=center, base=bottom]
  [list\\ \fbox{ \texttt{\{ x \} }}, align=center, base=bottom]

  [list\\ \fbox{ \texttt{\{ x \} }}, align=center, base=bottom
    [list\\ \fbox{ \texttt{\{ x \} }}, align=center, base=bottom]
    [oneItem\\ \fbox{ \texttt{\{ x \} }}, align=center, base=bottom]
  ]

  [decl\\ \fbox{ \texttt{ \{ \} } }, align=center, base=bottom
    [BOOL\\ ``\texttt{bool}'', align=center, base=bottom]
    [ID\\ ``\texttt{id}'', align=center, base=bottom]
    [SEMICOLON\\ ``\texttt{;}'', align=center, base=bottom]
  ]

  [stmt\\ \fbox{ \texttt{ \{ id \} } }, align=center, base=bottom
    [ID\\ ``\texttt{id}'', align=center, base=bottom]
    [ASSIGN\\ ``\texttt{=}'', align=center, base=bottom]
    [exp\\ \fbox{ \texttt{ \{ x \} }}, align=center, base=bottom]
    [SEMICOLON\\ ``\texttt{;}'', align=center, base=bottom]
  ]

  [stmt\\ \fbox{ \texttt{ \{ id \} } }, align=center, base=bottom
    [ID\\ ``\texttt{id}'', align=center, base=bottom]
    [ASSIGN\\ ``\texttt{=}'', align=center, base=bottom]
    [exp\\ \fbox{ \texttt{ \{ x \} }}, align=center, base=bottom]
    [SEMICOLON\\ ``\texttt{;}'', align=center, base=bottom]
  ]

  [exp\\ \fbox{ \texttt{ \{ id \} } }, align=center, base=bottom
    [exp\\ \fbox{ \texttt{ \{ id \} } }, align=center, base=bottom]
    [PLUS\\ ``\texttt{+}'', align=center, base=bottom]
    [exp\\ \fbox{ \texttt{ \{ id \} } }, align=center, base=bottom]
  ]

  [exp\\ \fbox{ \texttt{ \{ id \} } }, align=center, base=bottom
    [ID\\ ``\texttt{id}'', align=center, base=bottom]
  ]

  [exp\\ \fbox{ \texttt{ \{ \} } }, align=center, base=bottom
    [INTLIT\\ ``\texttt{val}'', align=center, base=bottom]
  ]

   
\end{forest}
\end{center}

                    
\end{document}

